% Options for packages loaded elsewhere
\PassOptionsToPackage{unicode}{hyperref}
\PassOptionsToPackage{hyphens}{url}
\PassOptionsToPackage{dvipsnames,svgnames,x11names}{xcolor}
%
\documentclass[
  letterpaper,
  DIV=11,
  numbers=noendperiod]{scrartcl}

\usepackage{amsmath,amssymb}
\usepackage{lmodern}
\usepackage{iftex}
\ifPDFTeX
  \usepackage[T1]{fontenc}
  \usepackage[utf8]{inputenc}
  \usepackage{textcomp} % provide euro and other symbols
\else % if luatex or xetex
  \usepackage{unicode-math}
  \defaultfontfeatures{Scale=MatchLowercase}
  \defaultfontfeatures[\rmfamily]{Ligatures=TeX,Scale=1}
\fi
% Use upquote if available, for straight quotes in verbatim environments
\IfFileExists{upquote.sty}{\usepackage{upquote}}{}
\IfFileExists{microtype.sty}{% use microtype if available
  \usepackage[]{microtype}
  \UseMicrotypeSet[protrusion]{basicmath} % disable protrusion for tt fonts
}{}
\makeatletter
\@ifundefined{KOMAClassName}{% if non-KOMA class
  \IfFileExists{parskip.sty}{%
    \usepackage{parskip}
  }{% else
    \setlength{\parindent}{0pt}
    \setlength{\parskip}{6pt plus 2pt minus 1pt}}
}{% if KOMA class
  \KOMAoptions{parskip=half}}
\makeatother
\usepackage{xcolor}
\usepackage[left=1in,right=1in,top=1in]{geometry}
\setlength{\emergencystretch}{3em} % prevent overfull lines
\setcounter{secnumdepth}{-\maxdimen} % remove section numbering
% Make \paragraph and \subparagraph free-standing
\ifx\paragraph\undefined\else
  \let\oldparagraph\paragraph
  \renewcommand{\paragraph}[1]{\oldparagraph{#1}\mbox{}}
\fi
\ifx\subparagraph\undefined\else
  \let\oldsubparagraph\subparagraph
  \renewcommand{\subparagraph}[1]{\oldsubparagraph{#1}\mbox{}}
\fi


\providecommand{\tightlist}{%
  \setlength{\itemsep}{0pt}\setlength{\parskip}{0pt}}\usepackage{longtable,booktabs,array}
\usepackage{calc} % for calculating minipage widths
% Correct order of tables after \paragraph or \subparagraph
\usepackage{etoolbox}
\makeatletter
\patchcmd\longtable{\par}{\if@noskipsec\mbox{}\fi\par}{}{}
\makeatother
% Allow footnotes in longtable head/foot
\IfFileExists{footnotehyper.sty}{\usepackage{footnotehyper}}{\usepackage{footnote}}
\makesavenoteenv{longtable}
\usepackage{graphicx}
\makeatletter
\def\maxwidth{\ifdim\Gin@nat@width>\linewidth\linewidth\else\Gin@nat@width\fi}
\def\maxheight{\ifdim\Gin@nat@height>\textheight\textheight\else\Gin@nat@height\fi}
\makeatother
% Scale images if necessary, so that they will not overflow the page
% margins by default, and it is still possible to overwrite the defaults
% using explicit options in \includegraphics[width, height, ...]{}
\setkeys{Gin}{width=\maxwidth,height=\maxheight,keepaspectratio}
% Set default figure placement to htbp
\makeatletter
\def\fps@figure{htbp}
\makeatother

\setkomafont{author}{\small}
\setkomafont{date}{\small}
\KOMAoption{captions}{tableheading}
\makeatletter
\makeatother
\makeatletter
\makeatother
\makeatletter
\@ifpackageloaded{caption}{}{\usepackage{caption}}
\AtBeginDocument{%
\ifdefined\contentsname
  \renewcommand*\contentsname{Table of contents}
\else
  \newcommand\contentsname{Table of contents}
\fi
\ifdefined\listfigurename
  \renewcommand*\listfigurename{List of Figures}
\else
  \newcommand\listfigurename{List of Figures}
\fi
\ifdefined\listtablename
  \renewcommand*\listtablename{List of Tables}
\else
  \newcommand\listtablename{List of Tables}
\fi
\ifdefined\figurename
  \renewcommand*\figurename{Figure}
\else
  \newcommand\figurename{Figure}
\fi
\ifdefined\tablename
  \renewcommand*\tablename{Table}
\else
  \newcommand\tablename{Table}
\fi
}
\@ifpackageloaded{float}{}{\usepackage{float}}
\floatstyle{ruled}
\@ifundefined{c@chapter}{\newfloat{codelisting}{h}{lop}}{\newfloat{codelisting}{h}{lop}[chapter]}
\floatname{codelisting}{Listing}
\newcommand*\listoflistings{\listof{codelisting}{List of Listings}}
\makeatother
\makeatletter
\@ifpackageloaded{caption}{}{\usepackage{caption}}
\@ifpackageloaded{subcaption}{}{\usepackage{subcaption}}
\makeatother
\makeatletter
\@ifpackageloaded{tcolorbox}{}{\usepackage[many]{tcolorbox}}
\makeatother
\makeatletter
\@ifundefined{shadecolor}{\definecolor{shadecolor}{rgb}{.97, .97, .97}}
\makeatother
\makeatletter
\makeatother
\ifLuaTeX
  \usepackage{selnolig}  % disable illegal ligatures
\fi
\IfFileExists{bookmark.sty}{\usepackage{bookmark}}{\usepackage{hyperref}}
\IfFileExists{xurl.sty}{\usepackage{xurl}}{} % add URL line breaks if available
\urlstyle{same} % disable monospaced font for URLs
\hypersetup{
  pdftitle={Misc. hypothesis testing practice},
  pdfauthor={Practice problems},
  colorlinks=true,
  linkcolor={blue},
  filecolor={Maroon},
  citecolor={Blue},
  urlcolor={Blue},
  pdfcreator={LaTeX via pandoc}}

\title{Misc. hypothesis testing practice}
\author{Practice problems}
\date{10/17/24}

\begin{document}
\maketitle
\ifdefined\Shaded\renewenvironment{Shaded}{\begin{tcolorbox}[borderline west={3pt}{0pt}{shadecolor}, breakable, frame hidden, enhanced, interior hidden, sharp corners, boxrule=0pt]}{\end{tcolorbox}}\fi

\begin{enumerate}
\def\labelenumi{\arabic{enumi}.}
\item
  For each of the statements (a) - (d), indicate if they are true or
  false interpretation of the following confidence interval. If false,
  provide or a reason or correction to the misinterpretation.

  ``You collect a large sample and calculate a 95\% confidence interval
  for the average number of cans of soda consumed annually per adult to
  be (440, 520), i.e.~on average, adults in the US consume just under
  two cans of soda per day''.

  \begin{enumerate}
  \def\labelenumii{\alph{enumii}.}
  \tightlist
  \item
    95\% of adults in the US consume between 440 and 520 cans of soda
    per year.
  \item
    There is a 95\% chance that the true population average per adult
    yearly soda consumption is between 440 and 520 cans.
  \item
    The true population average per adult soda consumption is between
    440 and 520 cans, with 95\% confidence.
  \item
    The average soda consumption of the people who were is sampled is
    between 440 and 520 cans of soda per year, with 95\% confidence.
  \end{enumerate}
\item
  A food safety inspector is called upon to investigate a restaurant
  with a few customer reports of poor sanitation practices. The food
  safety inspector uses a hypothesis testing framework to evaluate
  whether regulations are not being met. If the inspector determines the
  restaurant is in gross violation, its license to serve food will be
  revoked.

  \begin{enumerate}
  \def\labelenumii{\alph{enumii}.}
  \tightlist
  \item
    Write the hypotheses in words (no population parameters necessary).
  \item
    What is a Type I error in this context?
  \item
    What is a Type II error in this context?
  \item
    Which error is more problematic for the restaurant owner? For the
    diners? Why?
  \item
    Do you think the diners would prefer a higher or lower significance
    level \(\alpha\) compared to what the restaurant owner prefers?
    Explain.
  \end{enumerate}
\item
  Consider the following simple random sample
  \(x = (47, 4, 92, 47, 12, 8)\).

  Which of the following sets of values could be a possible bootstrap
  sample from the observe data above? If a set of values could not be a
  bootstrap sample, determine why not.

  \begin{enumerate}
  \def\labelenumii{\alph{enumii}.}
  \tightlist
  \item
    \((47, 47, 47, 47, 47, 47)\)
  \item
    \((92, 4, 13,8, 47, 4)\)
  \item
    \((4, 8, 12, 12, 47)\)
  \item
    \((12, 4, 8, 8, 92, 12)\)
  \item
    \((8, 47, 12, 12, 8, 4, 92)\)
  \end{enumerate}
\item
  For each of the following statements (a)-(e), indicate if they are a
  true or false interpretation of the p-value. If false, provide a
  reason or correction to the misinterpretation.

  ``You are wondering if the average amount of cereal in a 10 oz. cereal
  box is greater than 10 oz. You collect 50 boxes of cereal marketed as
  10 oz, conduct simulation-based hypothesis test, and obtain a p-value
  of 0.23.''

  \begin{enumerate}
  \def\labelenumii{\alph{enumii}.}
  \tightlist
  \item
    The probability that the average weight of all cereal boxes is 10
    oz. is 0.23.
  \item
    The probability that the average weight of all cereal boxes is
    something greater than 10 oz. is 0.23.
  \item
    Because the p-value is 0.23, the average weight of all cereal boxes
    is 10 oz.
  \item
    Because the p-value is small, the population average must be just
    barely about 10 oz.
  \item
    If \(H_{0}\) is true, the probability of observing another sample
    with an average as or more extreme as the data is 0.23.
  \end{enumerate}
\end{enumerate}



\end{document}
